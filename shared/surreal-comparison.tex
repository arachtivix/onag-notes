% Reusable component for analyzing surreal number comparisons
% This file provides a tabular format for step-by-step analysis of whether
% one surreal number is greater than or equal to another

% Required packages (should be loaded in main document)
% \usepackage{amsmath}
% \usepackage{array}  
% \usepackage{booktabs}

% Counter for comparison steps
\newcounter{comparisonStep}

% Define a new column type for centered math mode
\newcolumntype{C}{>{$}c<{$}}

% Simplified command for surreal number comparison analysis
% Usage: \surrealComparisonAnalysis{x_notation}{x_left}{x_right}{y_notation}{y_left}{y_right}{conclusion}
\newcommand{\surrealComparisonAnalysis}[7]{%
    \setcounter{comparisonStep}{0}%
    \begin{center}
    \textbf{Analysis of } $#1 \geq #4$
    \end{center}
    
    \noindent\textbf{Given:}
    \begin{align*}
    #1 &= \{#2|#3\} \\
    #4 &= \{#5|#6\}
    \end{align*}
    
    \noindent\textbf{To prove } $#1 \geq #4$\textbf{, we must show } $#4 \leq #1$\textbf{:}
    
    \vspace{0.5em}
    \noindent By definition, $#4 \leq #1$ if and only if:
    \begin{itemize}
        \item No element of $L_{#1}$ is $\geq #4$, \textbf{and}
        \item No element of $R_{#4}$ is $\leq #1$
    \end{itemize}
    
    \vspace{1em}
    \begin{tabular}{|c|l|l|c|}
    \hline
    \textbf{Step} & \textbf{Condition to Check} & \textbf{Analysis} & \textbf{Result} \\
    \hline
    \stepcounter{comparisonStep}%
    \thecomparisonStep & No $l \in L_{#1} = \{#2\}$ has $l \geq #4$ & 
    \begin{minipage}{5cm}
    \raggedright
    \surrealAnalyzeLeft{#2}{#4}
    \end{minipage} & 
    \surrealResultLeft{#2}{#4} \\
    \hline
    \stepcounter{comparisonStep}%
    \thecomparisonStep & No $r \in R_{#4} = \{#6\}$ has $r \leq #1$ & 
    \begin{minipage}{5cm}
    \raggedright
    \surrealAnalyzeRight{#6}{#1}
    \end{minipage} & 
    \surrealResultRight{#6}{#1} \\
    \hline
    \multicolumn{3}{|c|}{\textbf{Conclusion}} & #7 \\
    \hline
    \end{tabular}
    
    \vspace{1em}
}

% Helper commands for analysis
\newcommand{\surrealAnalyzeLeft}[2]{%
    \ifx\relax#1\relax%
        $L_{#1}$ is empty, so no elements to check%
    \else%
        \if\relax\detokenize{#1}\relax%
            $L$ is empty, so no elements to check%
        \else%
            Check if each element in $\{#1\}$ is $\geq #2$%
        \fi%
    \fi%
}

\newcommand{\surrealAnalyzeRight}[2]{%
    \ifx\relax#1\relax%
        $R$ is empty, so no elements to check%
    \else%
        \if\relax\detokenize{#1}\relax%
            $R$ is empty, so no elements to check%
        \else%
            Check if each element in $\{#1\}$ is $\leq #2$%
        \fi%
    \fi%
}

% Result commands
\newcommand{\surrealResultLeft}[2]{%
    \ifx\relax#1\relax%
        $\checkmark$%
    \else%
        \if\relax\detokenize{#1}\relax%
            $\checkmark$%
        \else%
            See details%
        \fi%
    \fi%
}

\newcommand{\surrealResultRight}[2]{%
    \ifx\relax#1\relax%
        $\checkmark$%
    \else%
        \if\relax\detokenize{#1}\relax%
            $\checkmark$%
        \else%
            See details%
        \fi%
    \fi%
}